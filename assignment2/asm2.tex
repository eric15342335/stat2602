\documentclass{article}
\usepackage{amssymb, amsthm, enumitem, microtype}
\usepackage[utf8]{inputenc}
\usepackage[margin=1in,top=0.6in,bottom=0.6in]{geometry}
\usepackage[fleqn]{amsmath}

\title{STAT2602 Assignment 2}
\author{Cheng Ho Ming, Eric (3036216734) [Section 1A, 2024]}

\begin{document}
\maketitle

\begin{enumerate}
% Question 1
\item
    \begin{enumerate}[label=(\roman*)]
    \item The probability density function (p.d.f.) for the uniform distribution $U[-\alpha, 0]$ is:
    \[
    f(x; \alpha) = \frac{1}{\alpha} \mathbb{I}_{[-\alpha, 0]}(x),
    \]
    where $\mathbb{I}_{[-\alpha, 0]}(x)$ is the indicator function, which is defined as:
    \[
    \mathbb{I}_{[-\alpha, 0]}(x) =
    \begin{cases}
    1 & \text{if } x \in [-\alpha, 0], \\
    0 & \text{otherwise}.
    \end{cases}
    \]

    The likelihood function for the sample $X_1, X_2, \dots, X_n$ is:
    \[
    L(\alpha) = \prod_{i=1}^{n} \frac{1}{\alpha} \mathbb{I}_{[-\alpha, 0]}(X_i).
    \]
    This product implies that the likelihood is zero if any $X_i$ lies outside the interval $[-\alpha, 0]$. \\
    Therefore, for the likelihood to be non-zero, all $X_i$ must lie in $[-\alpha, 0]$, i.e., $-\alpha \leq X_i \leq 0$ for all $i$.

    Hence, the likelihood function becomes:
    \[
    L(\alpha) = \frac{1}{\alpha^n} \mathbb{I}_{[-\alpha, 0]}(\min(X_1, \dots, X_n)) \mathbb{I}_{[-\alpha, 0]}(\max(X_1, \dots, X_n)).
    \]

    The log-likelihood function is:
    \[
    \ell(\alpha) = -n \log(\alpha) + \log \left( \mathbb{I}_{[-\alpha, 0]}(\min(X_1, \dots, X_n)) \right) + \log \left( \mathbb{I}_{[-\alpha, 0]}(\max(X_1, \dots, X_n)) \right).
    \]

    From the log-likelihood function, we can observe that it becomes larger when $\alpha$ is smaller.

    Also, the indicator function must not return zero, as $\log(0)$ is undefined. \\
    Therefore, $\min(X_1, \dots, X_n) \geq -\alpha$ and $\max(X_1, \dots, X_n) \leq 0$ must be satisfied when we maximizing the likelihood by finding the smallest value of $\alpha$.

    As a result:
    \[
    \min(X_1, \dots, X_n) \geq -\alpha
    \quad \Rightarrow \quad
    \alpha \geq - \min(X_1, \dots, X_n).
    \]

    Therefore, the MLE of $\alpha$ is:
    \[
    \hat{\alpha} = -\min(X_1, \dots, X_n).
    \]

    \item The likelihood function is:
    \[
    L(\alpha) = \frac{1}{\alpha^n} \mathbb{I}_{[-\alpha, 0]}(\min(X_1, \dots, X_n)) \mathbb{I}_{(-a, 0]}(\max(X_1, \dots, X_n)).
    \]

    By the \textbf{factorization theorem}, a sufficient statistic for \(\alpha\) can be found by factorizing the likelihood function into two parts: one that depends on \(\alpha\) and another that does not depend on \(\alpha\). Applying this theorem, we can write the likelihood function as:

    \[
    g(T(X); \alpha) = \frac{1}{\alpha^n} \mathbb{I}_{[-\alpha, 0]}(\min(X_1, \dots, X_n)) \quad \quad
    h(X_1, \dots, X_n) = \mathbb{I}_{(-a, 0]}(\max(X_1, \dots, X_n))
    \]

    Therefore, the likelihood depends on \(\alpha\) only through \( \min(X_1, \dots, X_n) \), \\
    meaning that \( \min(X_1, \dots, X_n) \) is a sufficient statistic for \(\alpha\).
    \end{enumerate}

% Answer to Question 2
\item
The MLE of $\theta$ is the value of $\theta$ that maximizes $f(x; \theta)$ for the observed $x$. For a given observation $x = x_{\text{obs}}$,

\begin{itemize}
    \item If $x_{\text{obs}} = 0$ or $x_{\text{obs}} = 1$, then the MLE is $\hat{\theta} = 1$ because $\frac{1}{3} > \frac{1}{4}$ and $f(x; 3) = 0$.
    \item If $x_{\text{obs}} = 2$, then $\hat{\theta} = 2$ or $\hat{\theta} = 3$, because both give $f(x; \theta) = \frac{1}{4}$.
    \item If $x_{\text{obs}} = 3$, then $\hat{\theta} = 3$ because $\frac{1}{2}$ is the largest probability.
    \item If $x_{\text{obs}} = 4$, then $\hat{\theta} = 3$ because $\frac{1}{4}$ is the largest probability.
\end{itemize}

% Answer to Question 3
\item
    \begin{enumerate}[label=(\roman*)]
    \item The likelihood function is:
    \[
    L(\theta) = \prod_{i=1}^{n} \frac{\theta}{X_i^2} \mathbb{I} (0 < \theta \leq X_i < \infty)
    \]

    The log-likelihood function is:
    \begin{align*}
    \ell(\theta) &= \log(\frac{\theta^n \mathbb{I} (0 < \theta \leq X_1, \dots, X_n < \infty)}{\prod_{i=1}^{n} X_i^2 }) \\
    &= n \log(\theta) + \log(\mathbb{I} (0 < \theta \leq \min(X_1, \dots, X_n) < \infty)) - \log(\prod_{i=1}^{n} X_i^2) \\
    &= n \log(\theta) + \log(\mathbb{I} (0 < \theta \leq \min(X_1, \dots, X_n) < \infty)) - 2 \sum_{i=1}^{n} \log(X_i)
    \end{align*}

    Taking the gradient w.r.t. $\theta$:
    \[
    \frac{\partial \ell}{\partial \theta} = \frac{n}{\theta} > 0 \quad (\text{for } 0 < \theta \leq \min(X_1, \dots, X_n) < \infty)
    \]
    Thus, the likelihood is increasing function w.r.t. $\theta$. \\
    The MLE is the maximum value of $\theta$ that satisfies the constraint $0 < \theta \leq \min(X_1, \dots, X_n) < \infty$, which is $\hat{\theta} = \min(X_1, \dots, X_n)$.

    \item The expectation of $X_1^{1/3}$ is:
    \[
    E(X_1^{1/3}) = \int_{\theta}^{\infty} x^{1/3} \frac{\theta}{x^2} \, dx = \int_{\theta}^{\infty} \theta x^{-5/3} \, dx = \theta \left[ -\frac{3}{2} x^{-2/3} \right]_{\theta}^{\infty} = \theta (0 - (-\frac{3}{2} \theta^{-2/3})) = \frac{3}{2} \theta^{1/3}
    \]

    \item
    Since the expectation of $X$:
    \[
    E(X) = \int_{\theta}^{\infty} x \frac{\theta}{x^2} \, dx = \int_{\theta}^{\infty} \theta x^{-1} \, dx = \theta \left[ -\log(x) \right]_{\theta}^{\infty} = \theta (-\log(\infty) - (-\log(\theta)))
    \]
    diverges, we need to use $E(X_1^{1/3})$ for methods of moment estimator (MME).

    From (ii), we have $E(X_1^{1/3}) = \frac{3}{2} \theta^{1/3}$.
    By equating $E(X_1^{1/3})$ to the 1/3-th sample moment of $X$ $m_{1/3} = \frac{1}{n} \sum_{i=1}^{n} X_i^{1/3}$, we get:
    \[
    \frac{3}{2} \theta^{1/3} = m_{1/3} \quad \Rightarrow \quad (\frac{3}{2})^{3} \theta = m_{1/3}^3 \quad \Rightarrow \quad \theta = (\frac{2}{3})^{3} m_{1/3}^3 = \hat{\theta}_{MME}
    \]

    As $n \to \infty$, the 1/3-th sample moment of $X$ converges to $E(X_1^{1/3}) = \frac{3}{2} \theta^{1/3}$. Therefore,

    \[
    \hat{\theta}_{MME} = (\frac{2}{3})^{3} m_{1/3}^3 \to (\frac{2}{3})^{3} E(X_1^{1/3})^3 = \frac{2}{3}^3 (\frac{3}{2} \theta^{1/3})^3 = \theta
    \]

    Hence, $\hat{\theta}_{MME} \to_p \theta$. \\
    $\therefore$ The MME is consistent.

    \end{enumerate}

% Answer to Question 4
\item
    \begin{enumerate}[label=(\roman*)]
    \item The likelihood function is:
    \[
    L(p) = \prod_{i=1}^{n} p(1 - p)^{X_i} = p^n (1 - p)^{\sum_{i=1}^{n} X_i}.
    \]
    By the factorization theorem, $T = \sum_{i=1}^{n} X_i$ is a sufficient statistic for $p$.

    The p.d.f. can be rewritten as:
    \[
    f(x; p) = \exp\left( x \ln(1 - p) + \ln(p) \right).
    \]
    Since the geometric distribution belongs to the exponential family and the parameter space \(0 < p < 1\) is large enough, \(T = \sum_{i=1}^n X_i\) is also complete for \(p\).

    Therefore, the statistic \(T = \sum_{i=1}^n X_i\) is both sufficient and complete for \(p\).

    \item
    % To find the UMVUE of \( p \), we use the Lehmann-Scheffé theorem, which states that any unbiased estimator that is a function of a complete and sufficient statistic is the UMVUE.

    Given from (i), \( T = \sum_{i=1}^{n} X_i \) is a complete and sufficient statistic for \( p \).
    \begin{align*}
        E(X_1) &= \sum_{x=0}^{\infty} x \cdot p(1-p)^x = p \sum_{x=0}^{\infty} x(1-p)^x \\
        &= p \sum_{x=0}^{\infty} (x+1-1)(1-p)^x = p \left( \sum_{x=0}^{\infty} (x+1)(1-p)^x - \sum_{x=0}^{\infty} (1)(1-p)^x \right) \\
        &= p \left( \sum_{x=1}^{\infty} x(1-p)^{x-1} - \sum_{x=0}^{\infty} (1-p)^x \right) \\
        & \text{Since } \sum_{x=1}^{\infty} xp(1-p)^{x-1} \text{ is the expectation of a geometric distribution which is } \frac{1}{p}, \\
        &= p ( \frac{1}{p^2} - \frac{1}{1-(1-p)} ) = \frac{1}{p} - 1 = \frac{1-p}{p}
    \end{align*}
    As $X_1, X_2, \dots, X_n$ are i.i.d., the expectation of \( T \) is:
    \begin{align*}
    \because E(T) &= E\left( \sum_{i=1}^{n} X_i \right) = \sum_{i=1}^{n} E(X_i) = n \cdot E(X_1) = n \cdot \frac{1 - p}{p} \\
    \therefore E(T) &= n \cdot (\frac{1}{p} - 1) \Rightarrow \frac{E(T)}{n}+1 = \frac{1}{p} \Rightarrow \frac{E(T)+n}{n} = \frac{1}{p} \Rightarrow \frac{n}{E(T)+n} = p
    \end{align*}

    % The theorem 3.2 is called Lehmann-Scheffe theorem, but the lecture notes seems didn't state it.
    $\because E(\frac{n}{T+n}) = \frac{E(n)}{E(T)+E(n)} = \frac{n}{E(T)+n}=p$,

    By Theorem 3.2 in the lecture notes, $\frac{n}{T+n}$ is the UMVUE of $p$.
    \end{enumerate}

% Answer to Question 5
\item
    \begin{enumerate}[label=(\roman*)]
    \item Since $X_i \sim N(\frac{p}{q}, \sigma_1^2)$ and $Y_i \sim N(q, \sigma_2^2)$, the expectation of $T_1$ is:
    \[
    E(T_1) = \frac{1}{n} \sum_{i=1}^n E(X_i Y_i) = \frac{1}{n} \sum_{i=1}^n \frac{p}{q}q = p
    \]
    Therefore, $T_1$ is an unbiased estimator of $p$.

    \item Since $X$ and $Y$ are independent,
    \begin{align*}
    Var(T_1) &= \frac{1}{n^2} \sum_{i=1}^n Var(X_i Y_i) = \frac{1}{n^2} \sum_{i=1}^n \left(E(X_i^2) E(Y_i^2) - p^2 \right) = \frac{1}{n} (\sigma_1^2 + \frac{p^2}{q^2}) (\sigma_2^2 + q^2) - \frac{p^2}{n} \\
    &= \frac{1}{n} \left( \sigma_1^2 \sigma_2^2 + \sigma_1^2 q^2 + \frac{p^2 \sigma_2^2}{q^2}\right)
    \end{align*}

    \item Since $X$ and $Y$ are independent,
    \begin{align*}
    E(T_2) &= E\left( \left( \frac{1}{n} \sum_{i=1}^n X_i \right) \left( \frac{1}{n} \sum_{i=1}^n Y_i \right) \right) = \left(\frac{1}{n} \sum_{i=1}^n E(X_i) \right) \left( \frac{1}{n} \sum_{i=1}^n E(Y_i) \right) = \frac{p}{q} \cdot q = p
    \end{align*}
    Hence, $T_2$ is also an unbiased estimator of $p$.

    \item By the weak law of large numbers,
    \begin{align*}
        E(T_2) &= E\left( \left( \frac{1}{n} \sum_{i=1}^n X_i \right) \left( \frac{1}{n} \sum_{i=1}^n Y_i \right) \right) = E(\bar{X}) E(\bar{Y}) \\
        & \rightarrow_p \frac{p}{q} \cdot q = p \text{ as } n \rightarrow \infty
    \end{align*}
    $\therefore T_2$ is a consistent estimator of $p$.

    \item When $p = 0$ and $q^2 = \frac{\sigma_2^2}{n}$,
    \begin{align*}
    Var(T_1) &= \frac{1}{n} \left( \sigma_1^2 \sigma_2^2 + \sigma_1^2 q^2 + \frac{p^2 \sigma_2^2}{q^2}\right) = \frac{1}{n} \left( \sigma_1^2 \sigma_2^2 + \sigma_1^2 \frac{\sigma_2^2}{n} \right) = \frac{\sigma_1^2 \sigma_2^2}{n} + \frac{\sigma_1^2 \sigma_2^2}{n^2} \\
    Var(T_2) &= Var(\bar{X} \bar{Y}) = E(\bar{X}^2) E(\bar{Y}^2) - 0 = (Var(\bar{X}) + E(\bar{X})^2) (Var(\bar{Y}) + E(\bar{Y})^2) \\
    % I don't really know how to calculate the X_Bar stuff so I just use WLLN, but I think it's wrong to use WLLN here.
    & \rightarrow_p (\sigma_1^2)(\sigma_2^2 + \frac{\sigma_2^2}{n}) = \sigma_1^2 \sigma_2^2 + \frac{\sigma_1^2 \sigma_2^2}{n}
    \end{align*}
    When $n=1$, $T_1$ and $T_2$ has the same efficiency. \\
    When $n>1$, $T_1$ is more efficient than $T_2$.
    \end{enumerate}

% Answer to Question 6
\item
    \begin{enumerate}[label=(\roman*)]
    \item The expectation of $\bar{X}$ and $\frac{n}{n-1} S^2$ are:
    \begin{align*}
    E(\bar{X}) &= E(\frac{1}{n} \sum_{i=1}^n X_i) = \frac{1}{n} \sum_{i=1}^n E(X_i) = \frac{1}{n} \cdot n \lambda = \lambda \\
    E\left( \frac{n}{n-1} S^2 \right) &= \frac{n}{n-1} E(S^2) = \frac{n}{n-1} E(\frac{1}{n} \sum_{i=1}^n (X_i - \bar{X})^2) = \frac{n}{n-1} \frac{1}{n} \sum_{i=1}^n E((X_i - \bar{X})^2) \\
    % The technique/trick here is just lecture slide 3.3.1 example 3.4 (cont.) - normal distribution.
    &= \frac{1}{n-1} \sum_{i=1}^n Var(X_i - \bar{X}) = \frac{1}{n-1} \sum_{i=1}^n Var(X_i - \frac{X_1 + \dots + X_n}{n}) \\
    &= \frac{1}{n-1} \sum_{i=1}^n Var(\frac{n-1}{n} X_i - \sum_{j \neq i} \frac{1}{n} X_j) \\
    &= \frac{1}{n-1} \sum_{i=1}^n \left( \left( \frac{n-1}{n} \right)^2 Var(X_i) + \left( \frac{1}{n} \right)^2 \sum_{j \neq i} Var(X_j) \right) \\
    &= \frac{1}{n-1} \left( \left( \frac{n-1}{n} \right)^2 n \lambda + \left( \frac{1}{n} \right)^2 (n)(n-1) \lambda \right) \\
    &= \frac{n-1}{n} \lambda + \frac{1}{n} \lambda = \lambda
    \end{align*}
    Both are unbiased estimators of $\lambda$.

    \item Since $X \sim \text{Poisson}(\lambda)$ is from a exponential family, and the parameter $\lambda > 0$ contains an open set in $\mathbb{R}$, \\
    the sufficient and complete statistic for $\lambda$ would be $T = \sum_{i=1}^n X_i$.

    \item The second derivative of the log-p.m.f. is:
    \begin{align*}
    f(x; \lambda) &= \frac{e^{-\lambda} \lambda^x}{x!} \\
    \log(f(x; \lambda)) &= -\lambda + x \log(\lambda) - \log(x!) \\
    % Idk, but lecture slides and tutorials seems to treat log() as "natural log" ln().
    \frac{\partial \log(f(x; \lambda))}{\partial \lambda} &= -1 + \frac{x}{\lambda} \\
    \frac{\partial^2 \log(f(x; \lambda))}{\partial \lambda^2} &= -\frac{x}{\lambda^2}
    \end{align*}
    Therefore, the Fisher information about $\lambda$ contained in data $X_1, \dots, X_n$ is:
    \[
    I_n(\lambda) = nI(\lambda) = -n E(\frac{\partial^2 log(f(X; \lambda))}{\partial \lambda^2}) = n\frac{E(X)}{\lambda^2} = \frac{n\lambda}{\lambda^2} = \frac{n}{\lambda}
    \]

    \item Using the Cramer-Rao Lower Bound (CRLB),
    \[
    Var(\hat{\lambda}) \geq \frac{1}{I_n(\lambda)} = \frac{\lambda}{n}
    \]

    \item The variance of $\bar{X}$ and $\frac{n}{n-1} S^2$ are:
    \begin{align*}
    Var(\bar{X}) &= Var(\frac{1}{n} \sum_{i=1}^n X_i) = \frac{1}{n^2} \sum_{i=1}^n Var(X_i) = \frac{1}{n^2} \cdot n \lambda = \frac{\lambda}{n} \\
    Var\left( \frac{n}{n-1} S^2 \right) &= \frac{n^2}{(n-1)^2} Var(S^2) = \frac{n^2}{(n-1)^2} Var(\frac{1}{n} \sum_{i=1}^n (X_i - \bar{X})^2) \\
    &= \frac{n^2}{(n-1)^2} \frac{1}{n^2} \sum_{i=1}^n Var(X_i - \bar{X}) \\
    &= \frac{1}{(n-1)^2} \sum_{i=1}^n Var(X_i - \bar{X}) = \frac{1}{(n-1)^2} \sum_{i=1}^n Var(X_i - \frac{X_1 + \dots + X_n}{n}) \\
    &= \frac{1}{(n-1)^2} \sum_{i=1}^n Var(\frac{n-1}{n} X_i - \sum_{j \neq i} \frac{1}{n} X_j) \\
    &= \frac{1}{(n-1)^2} \sum_{i=1}^n \left( \left( \frac{n-1}{n} \right)^2 Var(X_i) + \left( \frac{1}{n} \right)^2 \sum_{j \neq i} Var(X_j) \right) \\
    &= \frac{1}{(n-1)^2} \left( \left( \frac{n-1}{n} \right)^2 n \lambda + \left( \frac{1}{n} \right)^2 (n)(n-1) \lambda \right) \\
    &= \frac{1}{n} \lambda + \frac{n-1}{n} \lambda = \lambda
    \end{align*}
    Since $Var(\bar{X}) < Var\left( \frac{n}{n-1} S^2 \right)$, $\bar{X}$ is more efficient than $\frac{n}{n-1} S^2$. \\
    $\because \bar{X}$ also achieves the Cramer-Rao Lower Bound. \\
    $\therefore \bar{X}$ should be preferred as an estimator of $\lambda$.
    \end{enumerate}

% Answer to Question 7
\item
    \begin{enumerate}[label=(\roman*)]
    \item Let $\phi = (\theta, \theta^2)$. Denote the sample mean and sample variance as $\bar{X}$ and $S^2$ respectively. \\
    The joint p.d.f. of $X_1, X_2, \dots, X_n \sim N(\theta, \theta^2)$ is:
    \begin{align*}
    & f(x_1, x_2, \dots, x_n; \phi) \\
    =\ & \prod_{i=1}^n \frac{1}{\sqrt{2\pi\theta^2}} \exp\left( -\frac{(x_i - \theta)^2}{2\theta^2} \right) \\
    =\ & \frac{1}{(2\pi\theta^2)^{n/2}} \exp\left( -\frac{1}{2\theta^2} \sum_{i=1}^n (x_i - \theta)^2 \right) \\
    =\ & \frac{1}{(2\pi\theta^2)^{n/2}} \exp\left( -\frac{1}{2\theta^2} \left( \sum_{i=1}^n x_i^2 - 2\theta \sum_{i=1}^n x_i + n\theta^2 \right) \right) \\
    =\ & \frac{1}{(2\pi\theta^2)^{n/2}} \exp\left( -\frac{1}{2\theta^2} \left( \sum_{i=1}^n x_i^2 - 2\theta n\bar{X} + n\theta^2 \right) \right) \\
    =\ & \frac{1}{(2\pi\theta^2)^{n/2}} \exp\left( -\frac{1}{2\theta^2} (\sum_{i=1}^n x_i^2 - n\bar{X}^2 + n\bar{X}^2 - 2\theta n\bar{X} + n\theta^2) \right) \\
    =\ & \frac{1}{(2\pi\theta^2)^{n/2}} \exp\left( -\frac{1}{2\theta^2} (\sum_{i=1}^n x_i^2 - n\bar{X}^2) \right) \exp\left( -\frac{1}{2\theta^2} (n\bar{X}^2 + n\theta^2 - 2\theta n\bar{X}) \right) \\
    =\ & \frac{1}{(2\pi\theta^2)^{n/2}} \exp\left( -\frac{n}{2\theta^2} (\frac{1}{n} \sum_{i=1}^n x_i^2 - \bar{X}^2) \right) \exp\left( -\frac{n}{2\theta^2} (\bar{X} - \theta)^2 \right) \\
    =\ & \frac{1}{(2\pi\theta^2)^{n/2}} \exp\left( -\frac{n}{2\theta^2} S^2 \right) \exp\left( -\frac{n}{2\theta^2} (\bar{X} - \theta)^2 \right)
    \end{align*}
    By the factorization theorem, $T = (\bar{X}, S^2)$ is a sufficient statistic for $\phi$.
    % See ch.3 notes page 9 for answer.

    \item Let $\phi = (\theta, \theta^2) = (\theta, \xi)$ where $\xi = \theta^2$.
    The likelihood function based on the sample \(X_1, X_2, \dots, X_n\) is:
    \[
    L(\phi) = \prod_{i=1}^n \frac{1}{\sqrt{2\pi \xi}} \exp\left(-\frac{(X_i - \theta)^2}{2\xi}\right)
    \]
    Taking the logarithm of the likelihood:
    \[
    \ell(\phi) = \log L(\phi) = -\frac{n}{2} \log(2\pi \xi) - \frac{1}{2 \xi} \sum_{i=1}^n (X_i - \theta)^2
    \]
    Taking the derivative of the log-likelihood with respect to \(\theta\):
    \begin{align*}
    \frac{\partial \ell}{\partial \theta} &= \frac{1}{2 \xi} \sum_{i=1}^n 2(X_i - \theta) = \frac{1}{\xi} \sum_{i=1}^n (X_i - \theta) \\
    &= \frac{1}{\xi} \left( \sum_{i=1}^n X_i - n\theta \right)
    \end{align*}
    Letting it to zero and solve for \(\theta\):
    \begin{align*}
    \frac{1}{\xi} \left( \sum_{i=1}^n X_i - n\theta \right) &= 0 \\
    \sum_{i=1}^n X_i - n\theta &= 0 \\
    \theta &= \frac{1}{n} \sum_{i=1}^n X_i = \bar{X}
    \end{align*}
    Given the distribution is normal, the MLE is indeed maximum. \\
    Thus, the MLE of \(\theta\) is:
    \[
    \hat{\theta} = \bar{X} = \frac{1}{n} \sum_{i=1}^n X_i
    \]
    \item
    % Idk how to do this properly
    \begin{align*}
    E(\bar{X}) &= E\left( \frac{1}{n} \sum_{i=1}^n X_i \right) = \frac{1}{n} \sum_{i=1}^n E(X_i) = \frac{1}{n} \cdot n \theta = \theta \\
    Var(\bar{X}) &= Var\left( \frac{1}{n} \sum_{i=1}^n X_i \right) = \frac{1}{n^2} \sum_{i=1}^n Var(X_i) = \frac{1}{n^2} \cdot n \theta^2 = \frac{\theta^2}{n}
    \end{align*}
    By the Central Limit Theorem, as \( n \to \infty \), \( \bar{X} \sim_d N(\theta, \frac{\theta^2}{n}) \).
    \end{enumerate}

% Answer to Question 8
\item
    \begin{enumerate}[label=(\roman*)]
    \item As $m \geq 2$, $\mu_2 = s_1$, $X_2 \sim N(s_1, \sigma_2^2)$ and thus $E(X_2) = s_1$.
    \begin{align*}
    \tilde{s_1} &= \frac{X_1 + 2X_2}{3} \\
    E(\tilde{s_1}) &= \frac{E(X_1 + 2X_2)}{3} = \frac{E(X_1)+2E(X_2)}{3} \\
    &= \frac{s_1 + 2s_1}{3} = s_1
    \end{align*}
    As $E(\tilde{s_1}) = s_1$, $\tilde{s_1}$ is an unbiased estimator of $s_1$.

    \item The likelihood function of $X_1, X_2, \dots, X_{2m}$ is:
    \begin{align*}
    L(s_1, s_2) &= \prod_{i=1}^{m} \frac{1}{\sqrt{2\pi \sigma_i^2}} \exp\left( -\frac{(X_i - s_1)^2}{2\sigma_i^2} \right) \prod_{i=m+1}^{2m} \frac{1}{\sqrt{2\pi \sigma_i^2}} \exp\left( -\frac{(X_i - s_2)^2}{2\sigma_i^2} \right) \\
    &= \prod_{i=1}^{n} \frac{1}{\sqrt{2\pi \sigma_i^2}} \prod_{i=1}^{m} \exp\left( -\frac{(X_i - s_1)^2}{2\sigma_i^2} \right) \prod_{i=m+1}^{2m} \exp\left( -\frac{(X_i - s_2)^2}{2\sigma_i^2} \right) \\
    \end{align*}
    Taking the logarithm of the likelihood:
    \begin{align*}
    \ell(s_1, s_2) &= \log L(s_1, s_2) \\
    &= \sum_{i=1}^n \log \left( \frac{1}{\sqrt{2\pi \sigma_i^2}} \right) - \sum_{i=1}^{m} \frac{(X_i - s_1)^2}{2\sigma_i^2} - \sum_{i=m+1}^{2m} \frac{(X_i - s_2)^2}{2\sigma_i^2} \\
    \end{align*}
    Taking the derivative of the log-likelihood with respect to $s_1$ and $s_2$:
    \[
    \frac{\partial \ell}{\partial s_1} = \sum_{i=1}^{m} \frac{X_i - s_1}{\sigma_i^2} \quad \text{and} \quad \frac{\partial \ell}{\partial s_2} = \sum_{i=m+1}^{2m} \frac{X_i - s_2}{\sigma_i^2}
    \]
    Letting them to zero and solve for $s_1$ and $s_2$:
    \begin{align*}
    \sum_{i=1}^{m} \frac{X_i - s_1}{\sigma_i^2} &= 0 \Rightarrow \sum_{i=1}^{m} \frac{X_i}{\sigma_i^2} = \sum_{i=1}^{m} \frac{s_1}{\sigma_i^2} \Rightarrow s_1 = \frac{\sum_{i=1}^{m} \frac{X_i}{\sigma_i^2}}{\sum_{i=1}^{m} \frac{1}{\sigma_i^2}} \\
    \sum_{i=m+1}^{2m} \frac{X_i - s_2}{\sigma_i^2} &= 0 \Rightarrow \sum_{i=m+1}^{2m} \frac{X_i}{\sigma_i^2} = \sum_{i=m+1}^{2m} \frac{s_2}{\sigma_i^2} \Rightarrow s_2 = \frac{\sum_{i=m+1}^{2m} \frac{X_i}{\sigma_i^2}}{\sum_{i=m+1}^{2m} \frac{1}{\sigma_i^2}}
    \end{align*}
    \[
    \therefore \hat{s_1} = \frac{\sum_{i=1}^{m} \frac{X_i}{\sigma_i^2}}{\sum_{i=1}^{m} \frac{1}{\sigma_i^2}} \quad \text{and} \quad \hat{s_2} = \frac{\sum_{i=m+1}^{2m} \frac{X_i}{\sigma_i^2}}{\sum_{i=m+1}^{2m} \frac{1}{\sigma_i^2}}
    \]

    \item After setting $\sigma_i^2 = \frac{m}{i}$, the MLEs of $s_1$ is:
    \begin{align*}
    \hat{s_1} &= \frac{\sum_{i=1}^{m} \frac{X_i}{\frac{m}{i}}}{\sum_{i=1}^{m} \frac{1}{\frac{m}{i}}} = \frac{\sum_{i=1}^{m} \frac{iX_i}{m}}{\sum_{i=1}^{m} \frac{i}{m}} = \frac{\sum_{i=1}^{m} iX_i}{\sum_{i=1}^{m} i} = \frac{\sum_{i=1}^{m} iX_i}{\frac{m(m+1)}{2}} = \frac{2}{m(m+1)} \sum_{i=1}^{m} iX_i
    \end{align*}
    Variance of $\hat{s_1}$:
    \begin{align*}
    Var(\hat{s_1}) &= Var\left( \frac{2}{m(m+1)} \sum_{i=1}^{m} iX_i \right) = \frac{4}{m^2(m+1)^2} \sum_{i=1}^{m} i^2 Var(X_i) = \frac{4}{m^2(m+1)^2} \sum_{i=1}^{m} i^2 \frac{m}{i} \\
    &= \frac{4}{m(m+1)^2} \sum_{i=1}^{m} i = \frac{4}{m(m+1)^2} \cdot \frac{m(m+1)}{2} = \frac{2}{m+1}
    \end{align*}
    Variance of $\tilde{s_1}$:
    \begin{align*}
    Var(\tilde{s_1}) &= Var\left( \frac{X_1 + 2X_2}{3} \right) = \frac{1}{9} Var(X_1 + 2X_2) = \frac{1}{9} (Var(X_1) + 4Var(X_2)) \\
    &= \frac{1}{9} (\sigma_1^2 + 4\sigma_2^2) = \frac{1}{9} \left( \frac{m}{1} + 4 \cdot \frac{m}{2} \right) = \frac{1}{9} \left( m + 2m \right) = \frac{3m}{9} = \frac{m}{3}
    \end{align*}
    Comparing their variances: \\
    At $m=2$, $Var(\hat{s_1}) = \frac{2}{3}$ and $Var(\tilde{s_1}) = \frac{2}{3}$. \\
    At $m>2$, $Var(\hat{s_1}) = \frac{2}{m+1} < \frac{m}{3} = Var(\tilde{s_1})$. \\
    $\therefore$ $\hat{s_1}$ is more efficient than $\tilde{s_1}$ for $m>2$. Otherwise, they are equally efficient. 

    \item For $\hat{s_1}$, the asymptotic distribution of $\hat{s_1}$ is:
    \[
    I(\hat{s_1}) = \frac{1}{m} I_m(s_1)
    \]
    \[
    = \frac{1}{m} E \left[ \left( \frac{d \ell(s_1)}{d s_1} \right)^2 \right]
    = \frac{1}{m} E \left[ \left( \sum_{i=1}^{m} \frac{i}{m(m+1)} (x_i - s_1) \right)^2 \right]
    \]
    \[
    = \frac{1}{m} E \left[ \left( \frac{2}{m(m+1)} \sum_{i=1}^{m} i (x_i - s_1) \right)^2 \right]
    \]
    \[
    = \frac{1}{m^3} \left[ \text{Var} \left( \sum_{i=1}^{m} i (x_i - s_1) \right) + E \left( \sum_{i=1}^{m} i (x_i - s_1) \right)^2 \right]
    \]
    \[
    = \frac{1}{m^3} \left[ \sum_{i=1}^{m} i^2 \text{Var}(x_i) + \sum_{i=1}^{m} i^2 E(x_i) E(x_i - s_1)^2 \right]
    \]
    \[
    = \frac{1}{m^3} \left[ \sum_{i=1}^{m} i^2 \frac{m}{i} + \sum_{i=1}^{m} i^2 E(x_i) \right]
    \]
    \[
    = \frac{\sum_{i=1}^{m} i}{m(m+1)} + o
    = \frac{m+1}{2m}
    \]
    Hence,
    \[
    \hat{s_1} \xrightarrow{d} N \left( s_1, \frac{2}{m+1} \right)
    \]

    Similarly,
    \[
    I_n(\hat{s_2}) = \frac{1}{m} I_m(s_2)
    \]
    \[
    = \frac{1}{m} E \left[ \left( \sum_{i=m+1}^{2m} \frac{i}{m(m+1)} (x_i - s_2) \right)^2 \right]
    \]
    \[
    = \frac{1}{m^3} \sum_{i=m+1}^{2m} i^2 \left( \text{Var}(x_i) + E(x_i) E(x_i - s_2)^2 \right)
    \]
    \[
    = \frac{1}{m^3} \left[ \sum_{i=m+1}^{2m} i^2 \frac{m}{i} \right]
    = \frac{3m+1}{2m}
    \]
    Hence,
    \[
    \hat{s_2} \xrightarrow{d} N \left( s_2, \frac{3m+1}{2m} \right)
    \]

    \item As $m \to \infty$,
    \begin{align*}
    \lim_{m \to \infty} [E(\hat{s_1}) - s_1] &= \lim_{m \to \infty} \left[ \frac{2}{m(m+1)} \sum_{i=1}^{m} iE(X_i) - s_1 \right] = 0 \\
    \lim_{m \to \infty} [E(\hat{s_2}) - s_2] &= \lim_{m \to \infty} \left[ \frac{\sum_{i=m+1}^{2m} \frac{iE(X_i)}{m}}{\sum_{i=m+1}^{2m} \frac{i}{m}} - s_2 \right] = 0
    \end{align*}
    $\therefore$ The MLE of $s_1$ and $s_2$ are (asymptotic) unbiased estimators of $s_1$ and $s_2$ respectively.

    As $m \to \infty$, $Var(\hat{s_1}) = \frac{2}{m+1} \to 0$ and $Var(\hat{s_2}) = \frac{m}{3} \to 0$. \\
    $\therefore$ The MLE of $s_1$ and $s_2$ are consistent estimators of $s_1$ and $s_2$ respectively.

    \end{enumerate}


% Answer to Question 9
\item
    \begin{enumerate}[label=(\roman*)]

    \item The likelihood function of $\theta$ based on $X_1, X_2, \dots, X_n$ is:
    \[
    L(\theta; X_1, \dots, X_n) = \prod_{i=1}^n \frac{\theta}{X_i^{\theta + 1}} = \theta^n \prod_{i=1}^n \frac{1}{X_i^{\theta + 1}}.
    \]

    \item For $x \geq 1$ , the joint p.d.f. can be rewritten as:
    \begin{align*}
    f(x_1, x_2, \dots, x_n; \theta) &= \prod_{i=1}^n \frac{\theta}{x_i^{\theta + 1}} = \theta^n \prod_{i=1}^n {x_i^{-(\theta + 1)}} \\
    &= \theta^n e^{-(\theta + 1) \sum_{i=1}^n \ln(x_i)} \\
    &= g(T(x_1, x_2, \dots, x_n), \theta) \ h(x_1, x_2, \dots, x_n) \text{ where } h(\dots) = 1
    \end{align*}
    Using the $\textbf{factorization theorem}$, we got a scalar sufficient statistic T:
    \[
    T = \sum_{i=1}^n \ln(X_i).
    \]

    \item The Fisher information \( I_n(\theta) \) is given by:

    \begin{align*}
    I_n(\theta) &= nI(\theta) = -nE\left[ \frac{\partial^2}{\partial \theta^2} \log f(X; \theta) \right] = -nE\left[ \frac{\partial^2}{\partial \theta^2} \log \left( \frac{\theta}{X^{\theta + 1}} \right) \right] \\
    &= -nE\left[ \frac{\partial^2}{\partial \theta^2} (\log(\theta) - (\theta + 1) \log(X)) \right] = -nE\left[ \frac{\partial}{\partial \theta} \left( \frac{1}{\theta} - \log(X) \right) \right] \\
    &= -nE\left[ -\frac{1}{\theta^2} \right] = \frac{n}{\theta^2}.
    \end{align*}

    \item The Cramer-Rao Lower Bound (CRLB) for estimating \( \theta \) is:
    \[
    \text{CRLB} = \frac{1}{I_n(\theta)} = \frac{\theta^2}{n}.
    \]

    \item From the log-likelihood:
    \[
    \log L(\theta) = n \log \theta - (\theta + 1) \sum_{i=1}^n \log X_i
    \]
    Taking the derivative with respect to \( \theta \) and letting it to zero:
    \[
    \frac{n}{\theta} - \sum_{i=1}^n \log X_i = 0 \quad \Rightarrow \quad \theta = \frac{n}{\sum_{i=1}^n \log X_i}
    \]
    To confirm it is a MLE, we take the second derivative with respect to \( \theta \):
    \[
    \frac{\partial^2 \log L(\theta)}{\partial \theta^2} = -\frac{n}{\theta^2} < 0 \text{ given } \theta > 0.
    \]
    $\because$ The likelihood function is concave (open downward), the MLE is the maximum. \\
    $\therefore$ The MLE of \( \theta \) is $\hat{\theta} = \frac{n}{\sum_{i=1}^n \log X_i}$.

    \item
    % Slide (Theorem 3.4)
    By Central Limit Theorem and Cramer-Rao Lower Bound, under regular conditions, the asymptotic distribution of the MLE is:
    \[
    \frac{\hat{\theta} - \theta}{\sqrt{1/I_n(\theta)}} \sim N(0, 1) \quad \Rightarrow \quad \hat{\theta} \sim N\left( \theta, \frac{\theta^2}{n} \right) \text{ as } n \to \infty.
    \] 
    \end{enumerate}

\end{enumerate}

\end{document}
